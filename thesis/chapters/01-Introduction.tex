\chapter{Introduction}
\label{Chapter-Introduction}

\section{Purpose and Motivation}
\hspace{2mm}This project began as an effort to integrate a user-friendly `environment' layer into Kubernetes—an infrastructure known 
for its power and scalability, yet lacking direct support for user-centric entities. The goal was to make 
an environment where individual users can perform analytical computing in their own `space', especially enabling
data analysts to work with large-scale datasets, by abstracting the complexity of orchestration and container management.
As the system evolved, the focus deviated slightly towards a fully modular full stack platform that supports multiple users, secures
access with custom authentication, and enables storage provisioning, sharing, and execution of batch jobs using familiar tools such
as DuckDB, Bash, Octave, and Pandas~\cite{mckinney-proc-scipy-2010}. The motivation became to deliver a system 
that combines the robustness of cloud-native
technologies with the ease of use of desktop analytical environments—bridging the gap between DevOps infrastructure and 
end-user data workflows.

Ultimately, amongst some services, the bar was set to implement already existing enterprise level technology stacks to 
more basic and developer friendly as in ease of access and deployment versions. It had been a personal aspiration and 
challenge to create some utilities that are scoped for development with a laconic tone while maintaining inspiration by others.

\newpage

\section{Problem Statement}

\hspace{2mm}While modern data platforms such as Apache Airflow or Spark 
address large-scale, enterprise-level orchestration needs, they are often complex, heavyweight, and 
not designed with ease of use or modularity in mind—especially for individual analysts or smaller 
teams working on self-contained workflows.

This thesis does not seek to compete with such platforms, nor to introduce fundamentally new orchestration paradigms. 
Instead, it focuses on designing and implementing a lightweight, extensible system that simplifies the execution of batch 
data analysis jobs in a multi-user environment.

The core problem addressed is the absence of a simple, user-accessible platform where authenticated users can:
\begin{itemize}
    \item Upload datasets to a common, structured storage layer,
    \item Submit analytical jobs using containerized tools such as DuckDB, Pandas, or Octave,
    \item View job outputs and logs through a unified interface,
    \item And share or manage their data and jobs in a collaborative way.
\end{itemize}

This project presents a custom application built on top of Kubernetes~\cite{google-kubernetes} that abstracts away infrastructure complexity, 
offering a modular foundation upon which analytical tools can be plugged and executed in a reproducible, isolated manner. 
The overarching goal is to improve the user experience in running batch data workflows—not by reimagining distributed computation, 
but by making its power more accessible and flexible for real-world analysis tasks.



\section{Scope of the Project}
\hspace{2mm}The scope of this thesis includes the design, development, and integration of a modular microservice-based platform tailored to Kubernetes. 
The system introduces:

\begin{itemize}
    \item A custom authentication and user management service (\texttt{Minioth}).
    \item A central orchestration service (\texttt{Uspace}) that defines and manages user environments.
    \item A lightweight virtual filesystem abstraction (\texttt{Fslite}) for managing user data and metadata.
    \item Integration with MinIO for physical object storage.
    \item A WebSocket-enabled frontend service that provides real-time interaction and job monitoring capabilities.
\end{itemize}

The system supports uploading, sharing, and executing batch jobs using containerized tools, while abstracting the complexity of 
Kubernetes operations from the end user.

Out of scope for this project are scheduling algorithms, dynamic autoscaling and support for third-party 
authentication (e.g., OAuth~\cite{oauth2-rfc6749}).

The system is implemented in Go~\cite{golang}, a statically typed, compiled programming language designed for simplicity and concurrency.

The backend for the Services is implemented using the Gin web framework for Go~\cite{gingonic}, which provides performant HTTP routing and middleware integration.

\section{Thesis Contributions}
\hspace{2mm}The primary contributions of this thesis are as follows:
\begin{itemize}
    \item The design and implementation of a fully functional user-centric platform deployed on Kubernetes, 
    abstracting core orchestration mechanics from the user.
    \item A custom authentication service (\texttt{Minioth}) implementing user/group-based access control with secure 
    token issuance and verification.
    \item The \texttt{Uspace} service that defines and manages user environments, enabling secure job submission and resource 
    isolation.
    \item The \texttt{Fslite} abstraction layer that models user data and virtual resources in a hierarchical structure, 
    backed by object storage (MinIO).
    \item A WebSocket-integrated frontend allowing users to interactively monitor jobs and access 
    their workspace in real time.
    \item A demonstration of how Kubernetes-native resources and batch execution can be adapted to serve multi-user analytical 
    workflows in an accessible, extensible system.
\end{itemize}



\newpage
\section{Thesis Outline}
% Todo: Fill chapter descriptions
\begin{itemize}
    \item \textbf{Chapter 2 – Background and Theoretical Foundations:}  
    An overview of the core technologies and concepts underpinning the system, including Kubernetes, microservices, authentication models, 
    containerized workloads, and object storage.

    \item \textbf{Chapter 3 – Related Work:}  
    A review of existing platforms and tools related to data processing, user environment provisioning, and cloud-native architectures. 
    It highlights their limitations and positions this work within the broader landscape.

    \item \textbf{Chapter 4 – System Design and Architecture:}  
    The Architectural blueprint of the system, detailing the interactions between core services such as Minioth, Uspace, Fslite, and the Frontend. 
    The design goals of modularity, scalability, and security are emphasized.

    \item \textbf{Chapter 5 – System Usage \& Execution Flow:}  
    Describes how the system operates from both user and administrative perspectives. It details the deployment setup, system access points, 
    request-response structure, and error feedback mechanisms.  
    End-to-end workflows are illustrated through UI examples, including the job submission lifecycle.  

    \item \textbf{Chapter 6 – Evaluation and Robustness:}  
    An evaluation of the system’s behavior under various conditions, including performance, scalability, fault tolerance, and security. 
    Observations and informal benchmarks are included where applicable..

    \item \textbf{Chapter 7 – Conclusions and Future Work:}  
    Summarizes the work, reflects on its outcomes and limitations, and discusses potential directions for further development 
    and enhancement of the system.
\end{itemize}


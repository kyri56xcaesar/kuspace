\chapter{Conclusions and Future Work}
\label{Chapter-Conclusions}

\section{Conclusions}

This thesis presented the design and development of a modular, microservice-oriented job execution system tailored for Kubernetes-based infrastructures. Through a combination of lightweight services, clearly defined APIs, and a user-focused frontend, the platform enables authenticated users to upload data, submit analytical jobs, and retrieve results in a secure and controlled environment.

The system was implemented entirely in Go, using technologies such as Gin for HTTP routing, MinIO for object storage, and Kubernetes for orchestrated execution. A pluggable job architecture and storage abstraction layer allow for flexibility and future adaptability. Even though a formal evaluation was not conducted, the system demonstrates real-world viability through functional integration and local deployments on Minikube and Docker Desktop.

\section{Future Work}

Despite its completeness as a prototype, several areas of the system lend themselves to meaningful expansion and improvement:

\begin{itemize}
    \item \textbf{Caching and Optimization:} Introducing caching mechanisms (e.g., Redis) could reduce redundant computations and disk I/O, improving performance for repeated access to commonly used files or job results.
    
    \item \textbf{Job Pipelines:} Extending the job model to support pipelined execution—where the output of one tool feeds directly into another—would allow users to build more complex analytical workflows natively within the platform.
    
    \item \textbf{Distributed Job Models:} More sophisticated orchestration strategies could be explored, enabling distributed and cooperative execution of job chains, possibly integrating with task queues or event-driven architectures like Kafka or NATS.
    
    \item \textbf{Application Generation Framework:} A meta-application that allows administrators or advanced users to define new containerized applications (with their input/output schema and logic) via a UI or DSL would significantly expand the platform’s usability and extensibility.
    
    \item \textbf{CLI Client Tool:} To complement the web-based frontend, a command-line interface (CLI) tool could be developed, enabling power users and automation scripts to interact with the system more efficiently.
    
    \item \textbf{Evaluation and Benchmarking:} A structured performance and scalability analysis remains an open task. Real-cluster deployments and load testing would help validate assumptions about the system’s robustness and identify further bottlenecks.
\end{itemize}

In conclusion, the system serves as a strong foundation for scalable, secure, and user-friendly batch processing on Kubernetes. The outlined future work suggests a wide potential for transforming it into a general-purpose analytical platform adaptable across domains.
